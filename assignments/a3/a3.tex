\documentclass[fontsize=11pt]{article}
\usepackage{amsmath}
\usepackage[utf8]{inputenc}
\usepackage[margin=0.75in]{geometry}

\title{CSC110 Assignment 3: Loops, Mutation, and Applications}
\author{Azalea Gui & Peter Lin}
\date{\today}

\begin{document}
\maketitle

\section*{Part 1: Text generation, uniformly random model}

\begin{enumerate}

\item[1.]
\begin{enumerate}
    \item[(a)]
    TODO: Complete the table below (you will need to add more rows).

    You may find this tutorial useful: https://www.overleaf.com/learn/latex/Tables.

    \begin{tabular}{|l|l|l|l|}
    \hline
    Iteration & \verb|word| & \verb|words| & \verb|word_frequencies|\\
    \hline
    0 & N/A & \texttt{[]} & \texttt{[]} \\
    1 & & & \\
    \hline
    \end{tabular}

    \item[(b)]
    Including a specific example as the doctest's expected output when a function is random isn't a good idea because the function's output will be different from the expected output each time it is executed. Since the doctest only verifies if the actual output matches the expected output, specifying a single random outcome as the expected output among many other possibilities will likely produce an error when running the test.

    \item[(c)]
    For example, you can use $\texttt{words = \{'Hello': 1\}}$ as the words dictionary. In this case, \texttt{generate\_text\_uniform(words, 5)} has only one possible outcome, which is \texttt{'Hello Hello Hello Hello Hello'}, so we can use that as our statement and expected output in the doctest.
\end{enumerate}

\item[2.]
Complete this part in the provided \texttt{a3\_part1.py} starter file.
Do \textbf{not} include your solution in this file.

\end{enumerate}

\newpage

\section*{Part 2: Text generation, One-Word Context Model}

\begin{enumerate}

\item[0.]
This question is not to be handed in.

\item[1.]
One-word context model:

\begin{verbatim}
# TODO: Write your one-word context model here as a Python value.
# Write each key-value pair on a separate line, like on the assignment handout.
{
    'Love': ['is', 'is'],
    'is': ['patient.', 'kind.', 'not'],
    'patient.': ['Love'],
    'kind.': ['It'],
    'It': ['does', 'does', 'is'],
    'does': ['not', 'not'],
    'not': ['envy.', 'boast.', 'proud.'],
    'envy.': ['It'],
    'boast.': ['It']
}
\end{verbatim}

\item[2.]
Complete this part in the provided \texttt{a3\_part2.py} starter file.
Do \textbf{not} include your solution in this file.

\end{enumerate}

\newpage

\section*{Part 3: Loops and Mutation Debugging Exercise}

\begin{enumerate}
\item[1.]
The test \texttt{test\_star\_wars} passed, and the tests 
\texttt{test\_legally\_blonde} and \texttt{test\_transformers} failed.

\item[2.]
TODO: Write your answer here (and remember to edit a3\_part3.py as well).

\item[3.]
TODO: Write your answer here.
\end{enumerate}

\section*{Part 4: Forest Fires}

\begin{enumerate}
\item[1.]
Complete this part in the provided \texttt{a3\_part4.py} starter file.
Do \textbf{not} include your solution in this file.

\item[2.]
Complete this part in the provided \texttt{a3\_part4\_tests.py} starter file.
Do \textbf{not} include your solution in this file.

\item[3.]

\begin{enumerate}
\item[a.]
TODO: Write your answer here.

\item [b.]
TODO: Write your answer here.

\item[c.]
TODO: Write your answer here.
\end{enumerate}

\end{enumerate}

\end{document}
